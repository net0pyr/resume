
\documentclass[letterpaper,12pt]{article}

%\usepackage{fontspec}
%\usepackage{fontawesome}
\usepackage{latexsym}
\usepackage[empty]{fullpage}
\usepackage{titlesec}
\usepackage{marvosym}
\usepackage[usenames,dvipsnames]{color}
\usepackage{verbatim}
\usepackage{enumitem}
\usepackage[hidelinks]{hyperref}
\usepackage{fancyhdr}
\usepackage[english, russian]{babel}
\usepackage{tabularx}
\input{glyphtounicode}
\usepackage{fontawesome5}


\pagestyle{fancy}
\fancyhf{} % clear all header and footer fields
\fancyfoot{}
\renewcommand{\headrulewidth}{0pt}
\renewcommand{\footrulewidth}{0pt}

\addtolength{\oddsidemargin}{-0.5in}
\addtolength{\evensidemargin}{-0.5in}
\addtolength{\textwidth}{1in}
\addtolength{\topmargin}{-.5in}
\addtolength{\textheight}{1.0in}

\urlstyle{same}

\raggedbottom
\raggedright
\setlength{\tabcolsep}{0in}

\titleformat{\section}{
  \vspace{-4pt}\scshape\raggedright\large
}{}{0em}{}[\color{black}\titlerule \vspace{-5pt}]

\pdfgentounicode=1

\newcommand{\resumeItem}[1]{
  \item\small{
    {#1 \vspace{-2pt}}
  }
}

\newcommand{\mytitles}[2]{% \mytitles{<main>}{<sub>}
  \noindent\makebox[\textwidth]{#1\hfill#2}%
}

\newcommand{\resumeSubheading}[4]{
  \vspace{-2pt}\item
    \begin{tabular*}{0.97\textwidth}[t]{l@{\extracolsep{\fill}}r}
      \textbf{#1} & #2 \\
      \textit{\small#3} & \textit{\small #4} \\
    \end{tabular*}\vspace{-7pt}
}

\newcommand{\resumeSubSubheading}[2]{
    \item
    \begin{tabular*}{0.97\textwidth}{l@{\extracolsep{\fill}}r}
      \textit{\small#1} & \textit{\small #2} \\
    \end{tabular*}\vspace{-7pt}
}

\newcommand{\resumeProjectHeading}[2]{
    \item
    \begin{tabular*}{0.97\textwidth}{l@{\extracolsep{\fill}}r}
      \small#1 & #2 \\
    \end{tabular*}\vspace{-7pt}
}

\newcommand{\resumeSubItem}[1]{\resumeItem{#1}\vspace{-4pt}}

\renewcommand\labelitemii{$\vcenter{\hbox{\tiny$\bullet$}}$}

\newcommand{\resumeSubHeadingListStart}{\begin{itemize}[leftmargin=0.15in, label={}]}
\newcommand{\resumeSubHeadingListEnd}{\end{itemize}}
\newcommand{\resumeItemListStart}{\begin{itemize}}
\newcommand{\resumeItemListEnd}{\end{itemize}\vspace{-5pt}}


\begin{document}


\begin{center}
    \textbf{\Huge \scshape Зыков Иван Евгеньевич} \\ \vspace{3pt}
    \mytitles{\faPhone \textbf{ +7(921)-070-90-63}}{\href{mailto:naviz2409@gmail.com}{\Letter \textbf{ naviz2409@gmail.com}}} \\ \vspace{3pt}
    \mytitles{\href{https://github.com/net0pyr}{\faGithub \textbf{ net0pyr}}}{\href{https://t.me/net0pyr}{\faTelegram \textbf{ net0pyr}}}
\end{center}


\section{Образование}
  \resumeSubHeadingListStart
    \resumeSubheading
      {Гимназия №24}{г. Архангельск}
      {}{2012 -- 2019}
    \resumeSubheading
      {Архангельский государственный лицей им. М. В. Ломоносова}{г. Архангельск}
      {Физико-математический класс}{2021 -- 2023}
    \resumeSubheading
      {Университет ИТМО}{г. Санкт-Петербург}
      {Системное и прикладное программное обеспечение}{2023 -- н. в.}
  \resumeSubHeadingListEnd


\section{Достижения}
  \resumeSubHeadingListStart

    \resumeSubheading
      {Шаг в будущее}{Победитель}
      {Математика}{}

    \resumeSubheading
      {Шаг в будущее}{Призер}
      {Программирование}{}

    \resumeSubheading
      {Региональный этап ВСОШ}{Победитель}
      {Информатика}{}

  \resumeSubHeadingListEnd

\section{Проекты}
    \resumeSubHeadingListStart
      \resumeProjectHeading
          {\href{https://github.com/net0pyr/prog_labs}{\textbf{Лабораторные работы по программированию} \faExternalLink}}{} \\ \vspace{10pt}

      \resumeSubheading
          {\href{https://github.com/net0pyr/zapisyakal}{\textbf{Мобильное приложения для подсчета каллорий и записи тренировок} \faExternalLink}}{}{В стадии разработки}{}
        
    \resumeSubHeadingListEnd


\section{Технические навыки}
 \begin{itemize}[leftmargin=0.15in, label={}]
    \small{\item{
     \textbf{Языки программирования}{: Java, Python, C++, SQL (PostgreSQL), Kotlin, Bash} \\
     \textbf{Инструменты}{: Git, Docker} \\
    }}
 \end{itemize}

\section{Любимые учебные дисциплины}
 \begin{itemize}[leftmargin=0.15in, label={}]
    \small{\item{
     \textbf{Компьютерные сети} \\
     \textbf{Линейная алгебра} \\
     \textbf{Дискретная математика} \\
     \textbf{Основы профессиональной деятельности}{: Unix и БЭВМ} \\
     \textbf{Программирование}{: ООП на Java/Kotlin} \\
    }}
 \end{itemize}

\end{document}
